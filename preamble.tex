\usepackage{lmodern}
\KOMAoptions{DIV=last}
\usepackage{mathtools}
\usepackage{amsthm, amsfonts, amssymb}
\usepackage{parskip}
\usepackage{microtype}
\usepackage{extdash}
\usepackage{booktabs}
\usepackage{tikz, pgfplots}
\pgfplotsset{compat=1.18}
\usepackage[svgnames, dvipsnames, table]{xcolor}
\usepackage{hyperref}
\hypersetup{
  pdfauthor   = {Edward Wang},
  pdftitle    = {Mathematics},
  pdfsubject  = {Mathematics},
  pdfkeywords = {analysis, probability, prob, group theory, algebra, linear algebra, mathematics}
}

\newtheorem{theorem}{Theorem}[section]
\newtheorem{axiom}{Axiom}[section]
\newtheorem{lemma}{Lemma}[section]
\newtheorem{proposition}{Proposition}[section]
\newtheorem{corollary}{Corollary}[section]
\theoremstyle{definition}
\newtheorem{example}{Example}[section]
\newtheorem{definition}{Definition}[section]
\newtheorem{remark}{Remark}[section]

\DeclarePairedDelimiter\abs{\lvert}{\rvert}
\DeclarePairedDelimiter\norm{\lVert}{\rVert}
\DeclarePairedDelimiter\ceil{\lceil}{\rceil}
\DeclarePairedDelimiter\floor{\lfloor}{\rfloor}
\DeclarePairedDelimiterX\inpr[2]{\langle}{\rangle}{#1, #2}
\DeclarePairedDelimiter\cyc{\langle}{\rangle}

\DeclareMathOperator{\id}{id}
\DeclareMathOperator{\GL}{GL}
\DeclareMathOperator{\SL}{SL}
\DeclareMathOperator{\Hom}{Hom}
\DeclareMathOperator{\sgn}{sgn}
\DeclareMathOperator{\lcm}{lcm}

\DeclareMathOperator{\prob}{\mathbb{P}}

\newcommand\N{\mathbb{N}}
\newcommand\Z{\mathbb{Z}}
\newcommand\Q{\mathbb{Q}}
\newcommand\R{\mathbb{R}}
\newcommand\C{\mathbb{C}}
\newcommand\F{\mathbb{F}}
\renewcommand\O{\varnothing}

\let\epsilon\varepsilon
\let\vec\mathbf
